\documentclass[a4paper]{article}

\usepackage[T1]{fontenc}
\usepackage[italian]{babel}
\usepackage{tikz}


\begin{document}


\section{Assonometria}

\textcolor{red}{Mettere la teoria che ho usato per definire la rotazione $\alpha$, $\beta$.}


\section{SVG}

\subsection{\texttt{viewBox}}

Il \texttt{viewBox} e` calcolato in base all'ingombro del disegno e di un offset calcolato in base alla diagonale dell'ingombro.

\begin{tikzpicture}[scale=2]
\draw[gray] (0,0) rectangle (3,2);
\draw (-0.5,-0.5) rectangle (3.5,2.5);
\draw[<->] (0,-0.2) -- node[below]{$w$} (3,-0.2);
\draw[<->] (3.2,0) -- node[right]{$h$} (3.2,2);
\draw[latex-latex] (0,2) -- node[above right]{$d$} (3,0);
\draw[latex-, gray] (0,2) -- (0.2,2.2) node[right]{$(x_{min},y_{min})$};
\draw[latex-, gray] (3,0) -- (3.7,-0.3) node[right]{$(x_{max},y_{max})$};
\draw[<->] (-0.5,0.5) -- node[above]{$\Delta$} (0,0.5);
\draw[<->] (2,2) -- node[right]{$\Delta$} (2,2.5);
\draw[gray] (1.5,2) node[below]{geometria};
\end{tikzpicture}

La larghezza $w$ e l'altezza $h$ sono calcolate in base alle coordinate massime e minime:
$$ w = x_{max} - x_{min} \qquad h = y_{max} - y_{min} $$
L'offset aggiuntivo $\Delta$ e' calcolato in base ad una percentuale $p$ della diagonale $d=\sqrt{w^2+h^2}$:
$$ \Delta = p \cdot d $$
La percentuale $p$ e' un parametro di input del programma. Il valore standard e' $p=0.05$ (5\%).


\subsection{Spessore linee}


\subsection{Nodi}





\end{document}